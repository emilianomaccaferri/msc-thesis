\chapter{Introduzione}
Il fulcro della ricerca è la collaborazione, un elemento chiave che permette di unire le più distinte competenze per affrontare questioni della più disparata natura per trovare soluzioni valide e innovative. In un ambiente di ricerca sempre più in simbiosi con le ultime tecnologie, è importante trovare validi strumenti che permettano di sfruttarle al meglio, rendendo il processo di ricerca quanto più proficuo possibile.
\newline
\newline
Nella ricerca che si basa sull’analisi di dati assistita da calcolatori, uno degli strumenti più diffusi è sicuramente JupyterLab, un ambiente computazionale che permette di utilizzare svariati linguaggi di programmazione a supporto delle proprie attività di ricerca, permettendo a ricercatrici e ricercatori di condividere i propri esperimenti e analisi in maniera estremamente facile e riproducibile.
\newline
Lo scopo di questa tesi è quello di proporre NextPyter, una piattaforma che si pone come obiettivo quello di rendere accessibile a chiunque l’utilizzo di uno strumento come JupyterLab, integrando funzioni come la condivisione di file, supporto alla multiutenza, sicurezza e semplicità d’uso, tutte dal proprio browser.
\newline
Sfruttando le ultime tecnologie di containerizzazione come Kubernetes e Docker, l'obiettivo di NextPyter è quello di creare un'interfaccia completamente platform agnostic, tramite la quale sarà possibile gestire il ciclo di vita dei vari Jupyter notebook che saranno presenti nel sistema. NextPyter sarà eventualmente accoppiato ad un sistema di gestione e condivisione file, Nextcloud, per permettere la collaborazione tra gruppi di ricerca in maniera particolarmente semplice.
\newline
Nel secondo capitolo verrà fatta una breve introduzione alle pratiche di ricerca mediante l'utilizzo di notebook computazionali e di come queste ultime possono essere integrate con piattaforme di \textit{storage cloud based} e non.
\newline
Nel terzo capitolo, oltre ad un breve excursus sulla versione \textit{legacy} di NextPyter, sarà descritta la progettazione della piattaforma NextPyter, dove verrà delineato lo schema tecnico che verrà seguito, poi, nel quarto capitolo, che commenterà e dettaglierà l'effettiva realizzazione della piattaforma.
\newline
Nel quinto capitolo verranno proposte le conclusioni e gli sviluppi futuri per la piattaforma.