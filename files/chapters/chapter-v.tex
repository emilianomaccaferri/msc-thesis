\chapter{Conclusioni e sviluppi futuri}
NextPyter è una piattaforma che mira a risolvere i problemi riguardanti la gestione e l'amministrazione di ambienti di sviluppo Jupyter, facilitando l'accesso a gruppi di ricerca alle risorse computazionali necessarie per i propri compiti.
\newline
Sfruttando tecnologie di containerizzazione come Kubernetes e Docker, NextPyter agisce come un vero e proprio framework a se stante, astraendo completamente la gestione del ciclo di vita del sistema e offrendo un'interfaccia via API HTTP che permette a qualsiasi client di interagire con la piattaforma in maniera completamente standard, sfruttando, eventualmente, OAuth2.0 per l'autenticazione delle richieste.
\newline
Sono state effettuate due pubblicazioni scientifiche\cite{nextpyter-work-1}\cite{nextpyter-work-2} riguardo NextPyter, a dimostrazione che il progetto è in continua evoluzione.

\section{Sviluppi futuri}
Gli sviluppi futuri includeranno sicuramente il supporto ad un'autenticazione "offline" delle richieste, liberandosi quindi del supporto dell'\textit{introspection endpoint} nell'approccio descritto in \ref{reverse-proxy-oauth}, in modo da garantire un'autenticazione più veloce e che genera meno traffico all'interno del sistema. Ciò che si vuole implementare, in particolare, è una verifica crittografica a livello di \textit{application gateway}, dato che i token che viaggiano nel sistema sono \textit{JWT}, sfruttando le \textit{JWK}\footnote{https://datatracker.ietf.org/doc/html/rfc7517} esposte da Keycloak.
\newline
Altre migliorie verranno integrate per quanto riguarda lo \textit{scheduling} dei pod, infatti è già in creazione il supporto per integrare pod che necessitano di essere GPU-accelerated, che verranno \textit{schedulati} su nodi specifici (con GPU, per l'appunto). Per ora si mira a supportare GPU NVIDIA, ma è già stata individuata una modalità per integrare anche nodi che montano GPU AMD, a supporto di cluster più eterogenei.